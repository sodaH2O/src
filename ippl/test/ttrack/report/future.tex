\section{Future Work}
\subsection{Improving the program}
The current model has several drawbacks, but there are possible solutions. Here is a list of some of them.
\subsubsection{Weighting the particles}
At the moment our model only calculates the Touschek Lifetime, but does not calculate how the Touschek Effect affects the beam. To implement this we could give each particle a weight $Q_i$ telling us, how much is left out of the macro particle (we are usually simulating less particles then there are in the real beam.) On each collision we will then reduce the number $Q_i$ by something depending on the cross section of the scattering $\sigma$. So for example:
\begin{equation} Q_i \rightarrow Q_i \cdot f(\sigma) \end{equation}
Here $f(\sigma)$ is a function of the scattering angle (and maybe other things). However one has to take into account that our simulation times are very small compared to the lifetime. The lifetime is in the hours and our simulation times in the microseconds. So probably the effect one will see is very small.
\subsubsection{Include Synchrotron Radiation, Non-linear Optics, residual gas scattering and Polarizations}
Synchrotron Radiation changes the shape of the beam, since the shape of the beam influences the lifetime, including this radiation will change the computed lifetime. The same reasoning applies to non-linear optics and residual gas scattering.\\
Beam Polarization will change the scattering cross section. Since our lifetime depends on this cross section including it will also change our lifetime.
\subsubsection{Use a gaussian random number generator}
As speculated in the section comparing the semianalytic to the analytic computation, using a gaussian random number generator will probably lead to much better convergence.
\subsubsection{Use a 6D Mesh}
At the moment we are using an analytical formula to determine the particle densities in phase space, depending on the Twiss Parameters. However we could also directly determine these densities from our simulation, using a mesh. However, this would mean having a mesh in the phase space, which is 6 dimensional.

\subsection{Investigate the different models}
As described in the simulation part, we have 3 different models to compute the Touschek Lifetime (analytic, semi-analytic and 2 particles). These are giving different results. It might be interesting to investigate why.

\subsection{Changing the model}
It would be nice to model the real motion of the particles, so making the model less Monte Carlo and more deterministic. This is what we first tried to do, but failed. So it is still left for the future.

\subsection{Variable momentum acceptance}
At the moment, we suppose the momentum acceptance is the same around the whole beam line. However this is not the case, so we might need to consider adding a variable momentum acceptance for the beam line, specified in the lattice file.

\section{Acknowledgments}
I would like to thank Andreas Adelmann and Andreas Streun for their great help and support. Additional thanks go to Michael B\"oge and Roman Geus for their help. Without their help, I would certainly not been able to finish this project, and also for allowing me to work with equipment like the Cray-XT3 ``Horizon'' or to be part of doing measurements using the SLS.
